\section{Project Volunteer
Information}\label{project-volunteer-information}

Thank you for volunteering your time to help me with my senior project.
In this document I will attempt to cover my expectations from you as
well as any resources that you might find useful while working on my
project. Please let me know if you have any questions; I'd be happy to
discuss the project with you!

\subsection{Expectations}\label{expectations}

Over the course of this project, you will be implementing a series of
programs that illustrate various aspects of systems programming: such as
memory management, pointer manipulation, and conditional compilation.
Each volunteer is expected to have

\begin{itemize}
\itemsep1pt\parskip0pt\parsep0pt
\item
  Some experience with C++
\item
  Very little experience with either D or Rust
\end{itemize}

\subsubsection{Error Logs}\label{error-logs}

For my project, I am interested in the types of errors that are common
to each of the languages you will be working in. Specifically, here are
the types of errors I am studying:

\begin{center}
\begin{tabular}[c]{@{}lp{9cm}@{}}
\toprule
Category & Description \\
\midrule
Syntax error & Any error caused by your program failing to parse. Could
be a typo in a variable name, forgetting to put braces,
etc.\tabularnewline
Logic error & Your program parses correctly, but it does not run as
intended due to human error. Off-by-one errors are a common example of a
logic error.\tabularnewline
Resource error & The incorrect use of memory management. For example,
your program dereferences a \lstinline!NULL! pointer or uses deallocated
memory.\tabularnewline
\bottomrule
\end{tabular}
\end{center}

In order to obtain data on the types of errors that you run into, I
would like you to keep a log of every error you run into while working
on each project. Each log should contain the type of error you ran into,
a brief description of the error, and whether the error was caught at
compile-time or run-time. Here's an example of a log:

\begin{longtable}[c]{@{}lll@{}}
\toprule
Category & Description & When caught\tabularnewline
\midrule
\endhead
Syntax error & Wrote ``sring'' instead of ``string'' &
Compile-time\tabularnewline
Logic error & Accidentally iterated one time too many &
Run-time\tabularnewline
Resource error & Used a freed pointer & Run-time\tabularnewline
\bottomrule
\end{longtable}

In short, if the compiler spits out an error, or you have to change your
code after you noticed something didn't work, log it! The error logs are
key to my study so please take care that they are accurate.

\paragraph{Rust Note}\label{rust-note}

Since Rust is rather unstable, you might receive warnings about
deprecated or unstable features. You may fix these, and you do
\emph{not} have to include them in your logs. However, if you are unsure
if an error should be included or not, please include it anyways.

\subsubsection{Implementation}\label{implementation}

Each volunteer will be selected to implement their series of programs in
D or Rust.

\paragraph{Resources}\label{resources}

In order to keep the learning environment as free of bias as possible,
the resources you may use to learn Rust and D should be limited to the
following sites (i.e., no StackOverflow, no IRC, etc.):

\subparagraph{Additional Help}\label{additional-help}

If you would like additional help, you may contact me in which case I
will decide to offer help at my own discretion. I will offer help with:

\begin{itemize}
\itemsep1pt\parskip0pt\parsep0pt
\item
  Algorithmic Questions
\item
  Clarification
\item
  Installing the languages
\item
  Locating information about the languages in the books or docs
\end{itemize}

I will not:

\begin{itemize}
\itemsep1pt\parskip0pt\parsep0pt
\item
  Look at your code
\item
  Help you solve compile errors
\end{itemize}

You are free to use whatever editor you wish. Please install a syntax
highlighting package for your editor; I know that packages exist for
Sublime Text, emacs, and vim. Please do \emph{not} install any
autocomplete packages, as these may skew the results of the study.

\paragraph{Development}\label{development}

There will be test cases provided in order for you to test your
implementation. However, there is no problem if your program does not
pass all cases. As long as you spend a significant amount of time on
each program (say, 90 minutes), it doesn't matter if your program
``works'' in the end.

I may use snippets of your code in my final paper, in which case your
work will be kept anonymous.

\subsection{Installation}\label{installation}

It's probably easiest to use your own computer to install each
language's compiler. However, if you have a Windows PC, to save yourself
some installation headache I recommend either dual-booting or running a
VM of a Linux distro such as Ubuntu, or using the lab Macs. I've set up
\lstinline!project! with the needed dependencies of each language as
well.

You may find language-specific installation guides in the
\href{./rust_resources.md}{Rust resources} and \href{./d_resources.md}{D
resources} documents.

\subsection{Programming}\label{programming}

You will be asked to implement the following programs (listed in order
from easiest to hardest).

\emph{Note:} While I don't expect you to handle \emph{all} errors that
your program might run into, if you run into exceptional conditions
(that is, empty input, invalid indices, etc.), please handle the errors
as you feel fit. This could include throwing an exception or returning
an \lstinline!Error! object.

\subsubsection{\texorpdfstring{Sentence Splitter
(\href{http://www.ling.gu.se/~lager/python_exercises.html}{source})}{Sentence Splitter (source)}}\label{sentence-splitter-sourcesentence-splitter}

Define a function capable of splitting a text into sentences. The
standard set of heuristics for sentence splitting includes (but isn't
limited to) the following rules:

Sentence boundaries occur at one of ``.'' (periods), ``?'' or ``!'',
except that

\begin{itemize}
\itemsep1pt\parskip0pt\parsep0pt
\item
  Periods followed by whitespace followed by a lower case letter are not
  sentence boundaries.
\item
  Periods followed by a digit with no intervening whitespace are not
  sentence boundaries.
\item
  Periods followed by whitespace and then an upper case letter, but
  preceded by any of a short list of titles are not sentence boundaries.
  Sample titles include Mr., Mrs., Dr., Jr.
\item
  Periods internal to a sequence of letters with no adjacent whitespace
  are not sentence boundaries (for example, www.aptex.com, or e.g).
\item
  Periods followed by certain kinds of punctuation (notably comma and
  more periods) are probably not sentence boundaries.
\end{itemize}

Write a function \lstinline!splitSentences(string input)! that, given a
string, prints each sentence on a separate line. I am only concerned
with the function, so you may pass your test input to the function in
any way, whether it be hard-coded, or through stdin, etc.

\emph{Note:} For simplicity, we will assume that we only have one space
between each sentence.

\paragraph{Skills covered}\label{skills-covered}

\begin{itemize}
\itemsep1pt\parskip0pt\parsep0pt
\item
  String manipulation
\item
  Branching
\end{itemize}

\subsubsection{Integer Linked List}\label{integer-linked-list}

Define a
\href{http://en.wikipedia.org/wiki/Linked_list\#Singly_linked_list}{linked
list} data structure that operates on integers. The linked list should
support:

\begin{itemize}
\itemsep1pt\parskip0pt\parsep0pt
\item
  Insertion in O(n) time.
\item
  Deletion in O(n) time.
\item
  Retrieval in O(n) time.
\item
  A method to retrieve the size.
\item
  A method to retreive the head of the list.
\item
  A method to retrieve the tail of the list.
\end{itemize}

The IntegerLinkedList class should implement the following Java-like
interface:

\begin{lstlisting}[language=Java]
/**
 * A linked list containing only integers.
 */
interface IntegerLinkedList {
  /**
   * Inserts an element into the linked list at the specified index,
   * shifting any elements already in its position down.
   */
  public void insert(int index, int element);

  /**
   * Removes an element from the specified index, shifting any elements
   * already in the list back up.
   * @return The element that was removed.
   */
  public int remove(int index);

  /**
   * Retrieves an element at the specified position
   */
  public int get(int index);

  /**
   * Returns the number of integers in the linked list.
   */
  public int size();

  /**
   * Returns the first element of the linked list.
   */
  public int head();

  /**
   * Returns the last element of the linked list.
   */
  public int tail();
}
\end{lstlisting}

\paragraph{Skills covered}\label{skills-covered-1}

\begin{itemize}
\itemsep1pt\parskip0pt\parsep0pt
\item
  Object-orientation
\item
  Pointer manipulation
\item
  Memory allocation
\end{itemize}

\subsubsection{Generic Array List}\label{generic-array-list}

Implement an array list class (also known as a dynamic array). Unlike
the linked list class, this class should support generic elements, like
ArrayList in Java. You should use templates (D) or generics (Rust) to
implement this class.

The array list should support:

\begin{itemize}
\itemsep1pt\parskip0pt\parsep0pt
\item
  Insertion in amortized O(1) time.
\item
  Deletion in O(n) time.
\item
  Retrieval in O(1) time.
\item
  A method to retrieve the size.
\end{itemize}

The ArrayList class should implement the following Java-like interface:

\begin{lstlisting}[language=Java]
/**
 * An array that dynamically resizes as elements are added.
 */
interface ArrayList<E> {
  /**
   * Inserts an element into the array at the specified index,
   * shifting any elements already in the list down.
   */
  public void insert(int index, E element);

  /**
   * Removes an element from the specified index, shifting any
   * elements past its position back up.
   * @return The element that was removed.
   */
  public E remove(int index);

  /**
   * Retrieves an element at the specified index.
   */
  public E get(int index);

  /**
   * Returns the number of elements in the array.
   */
  public int size();
}
\end{lstlisting}

\paragraph{Skills covered}\label{skills-covered-2}

\begin{itemize}
\itemsep1pt\parskip0pt\parsep0pt
\item
  Object-orientation
\item
  Generic programming
\item
  Memory allocation
\end{itemize}

\subsubsection{Parallel Mergesort}\label{parallel-mergesort}

Implement the
\href{http://en.wikipedia.org/wiki/Merge_sort\#Parallel_merge_sort}{parallel
mergesort algorithm}. This algorithm should use
\href{http://dlang.org/phobos/std_parallelism.html}{std.parallelism} (D)
or \href{http://doc.rust-lang.org/std/thread/}{std::thread} (Rust). You
should also implement a sequential threshold of 10, meaning that if the
number of elements in a subarray is less than 10, you should not fork a
new thread.

Your function should take in an array.

\paragraph{Skills covered}\label{skills-covered-3}

\begin{itemize}
\itemsep1pt\parskip0pt\parsep0pt
\item
  Recursion
\item
  Concurrency
\end{itemize}

\subsubsection{Brainfsck Interpreter}\label{brainfsck-interpreter}

Write an interpreter for the
\href{http://esolangs.org/wiki/Brainfuck}{brainfsck} programming
language (thankfully, you are not required to write any brainfsck
programs on your own).

Your interpreter should take in a file name as the first argument to the
program, and respond to standard input and output as specified in
\href{http://en.wikipedia.org/wiki/Brainfuck\#Commands}{this Wikipedia
article}. Hint: your interpreter may need to perform multiple passes on
the input to handle brackets.

\paragraph{Skills covered}\label{skills-covered-4}

\begin{itemize}
\itemsep1pt\parskip0pt\parsep0pt
\item
  File I/O
\item
  Pointer manipulation
\end{itemize}

\subsection{Completion}\label{completion}

Upon completion of each program, you may email your code and error log
to me or upload it to a source-hosting website such as GitHub or
BitBucket.

\subsection{Conclusion}\label{conclusion}

Thank you again for volunteering! If you would like to contact me, I may
be reached on Facebook, through email at acr02011@mymail.pomona.edu, or
through my phone at (314) 440-8830.
