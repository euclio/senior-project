\section{D Resources}\label{d-resources}

\subsection{Installation}\label{installation}

\href{http://dlang.org}{D} has a number of compiler implementations, but
we will be using \lstinline!dmd!, the reference compiler.

On Windows or Macs, you may install D by using the
\href{http://ftp.digitalmars.com/dinstaller.exe}{binary installer}.

On Linux, you can likely install D with your distro's package manager,
but it might not be in the official repositories. For example, on
Ubuntu, you will have to add the d-apt repository and install
\lstinline!dmd!.

\begin{lstlisting}[language=sh]
$ sudo wget \
    http://master.dl.sourceforge.net/project/d-apt/files/d-apt.list \
    -O /etc/apt/sources.list.d/d-apt.list
$ sudo apt-get update
$ sudo apt-get -y --allow-unauthenticated install \
    --reinstall d-apt-keyring && sudo apt-get update
$ sudo apt-get install dmd
\end{lstlisting}

If you're using \lstinline!project!, you'll have to build D from source.
I've written a \href{./install_d.sh}{shell script} to do this, as the
process is pretty involved.

\subsection{D Documents}\label{d-documents}

\begin{itemize}
\itemsep1pt\parskip0pt\parsep0pt
\item
  \href{http://dlang.org/intro.html}{D Reference}
\item
  \href{http://ddili.org/ders/d.en/index.html}{Programming in D}
\end{itemize}

You should read the following resources before getting started with D,
though you will likely find other sections helpful:

\begin{itemize}
\itemsep1pt\parskip0pt\parsep0pt
\item
  \href{http://ddili.org/ders/d.en/index.html}{All Chapters up to
  Strings}
\item
  \href{http://ddili.org/ders/d.en/lifetimes.html}{Lifetimes}
\item
  \href{http://ddili.org/ders/d.en/member_functions.html}{Member
  Functions}
\item
  \href{http://ddili.org/ders/d.en/pointers.html}{Pointers}
\item
  \href{http://ddili.org/ders/d.en/templates.html}{Templates}
\item
  \href{http://ddili.org/ders/d.en/parallelism.html}{Parallelism}
\item
  \href{http://ddili.org/ders/d.en/files.html}{Files}
\end{itemize}
