\section{Rust Resources}\label{rust-resources}

\subsection{Installation}\label{installation}

\href{https://github.com/rust-lang/rust}{Rust} is a very volatile
language, with breaking changes being brought into the master branch
almost every day. Most developers who use Rust use the nightly releases
which means that they must constantly update their code to deal with the
most recent breaking changes. Since I do not want to introduce any
additional overhead to the project, we will be sticking to the
\textbf{1.0.0-alpha.2} release of Rust. Please ensure that you use this
version in order to keep source compatibility during your time working
on the project.

You may install Rust using one of the 1.0.0-alpha.2 (\textbf{not
nightly}) binary installers
\href{http://www.rust-lang.org/install.html}{located here}.

If you have problems with the binary installers, you may install Rust
from source in the following way. If you are attempting to build Rust on
\lstinline!project!, you may skip step 1, as the dependencies are
already installed.

\begin{enumerate}
\def\labelenumi{\arabic{enumi}.}
\item
  Install dependencies

  Use homebrew or your distro's package manager to install

  \begin{itemize}
  \itemsep1pt\parskip0pt\parsep0pt
  \item
    \lstinline!g++! 4.7 or \lstinline!clang++! 3.7
  \item
    \lstinline!python! 2.6 or later (not 3.x)
  \item
    GNU \lstinline!make! 3.81 or later
  \item
    \lstinline!curl!
  \item
    \lstinline!git!
  \end{itemize}
\item
  Download and build Rust:

\begin{lstlisting}[language=sh]
$ git clone https://github.com/rust-lang/rust.git && cd rust
$ git checkout 1.0.0-alpha.2
$ ./configure --prefix=$PWD
$ make -j5 && make install
\end{lstlisting}

  This will take a while, on my laptop it took just over an hour to
  compile.
\item
  Add Rust to your path.

\begin{lstlisting}[language=sh]
echo "export PATH=\$PATH:$PWD/bin" >> ~/.bashrc
source ~/.bashrc
\end{lstlisting}

  Test that rust installed correctly.

\begin{lstlisting}[language=sh]
$ rustc --version
rustc 1.0.0-dev
\end{lstlisting}
\end{enumerate}

\subsection{Documentation}\label{documentation}

Note that he docs are pinned to the 1.0.0-alpha.2 release.

\begin{itemize}
\itemsep1pt\parskip0pt\parsep0pt
\item
  \href{http://doc.rust-lang.org/1.0.0-alpha.2/index.html}{Rust
  Documentation}
\item
  \href{http://rustbyexample.com/}{Rust by Example}
\end{itemize}

\emph{Note:} since Rust is changing so rapidly, there is a chance that
Rust by Example will be outdated. The official documentation and
associated book will be correct, however.

You should read the following resources before getting starting with
Rust, though you will likely find other sections helpful:

\begin{itemize}
\itemsep1pt\parskip0pt\parsep0pt
\item
  \href{http://doc.rust-lang.org/1.0.0-alpha.2/book/basic.html}{All of
  Chapter 2: Basics} (skip the installation chapter)
\item
  \href{http://doc.rust-lang.org/1.0.0-alpha.2/book/method-syntax.html}{Method
  Syntax} (classes)
\item
  \href{http://doc.rust-lang.org/1.0.0-alpha.2/book/pointers.html}{Pointers}
  and
  \href{http://doc.rust-lang.org/1.0.0-alpha.2/book/ownership.html}{Ownership}
\item
  \href{http://doc.rust-lang.org/1.0.0-alpha.2/book/generics.html}{Generics}
\item
  \href{http://doc.rust-lang.org/1.0.0-alpha.2/book/concurrency.html}{Concurrency}
\item
  \href{http://rustbyexample.com/file.html}{File I/O}
\end{itemize}
