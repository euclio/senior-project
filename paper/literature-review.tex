\documentclass[finalcopy,short]{srpaper}

\usepackage{listings}

\title{Senior Project Literature Review}
\author{Andy Russell}
\date{\today}
\advisor{Professor Kim Bruce}

\begin{document}
\frontmatter

\section{Introduction}

Systems programming is an extremely important part of computing today. The raw
speed and low-level access to hardware provided by languages of this nature are
necessary for embedded systems, networking, and gaming, and any number of other
applications. However, such power comes with inherent
danger\footnote{Throughout this paper, when I refer to a concept or feature as
    ``dangerous,'' I mean that it is error-prone or difficult to reason about,
    especially if the errors that may arise from its use are resistant to
    debugging.}.

Mistakes in managing memory can lead to runtime crashes, or worse: security
vulnerabilities such as buffer overflow or format string
attacks~\cite{Shahriar:2012:MPS:2187671.2187673}. While there are many tools
designed to allow programmers to catch such errors, the ideal solution would be
to eliminate the burden of manual memory management. Other languages such as
Java and C\#, inspired by this goal, have removed that need. However, due to
their dependence on a virtual machine, they have sacrificed performance and the
ability to interface with hardware~\cite{Alexandrescu:2010:DPL:1875434}. The
languages examined in this paper, D and Rust, do not use a virtual machine, and
instead take the speed and power of C++ while attempting to make code easier to
write both from an expressiveness and correctness standpoint.

\section{C++}

C++ is a language developed in the early '80s by Bjarne Stroustrup. C++
remains one of the most popular languages in the world. C++ The features
particularly relevant to this paper are the C++ Macro Preprocessor and its
memory management facilities.

The C++ macro preprocessor is an integral part of the language. Perhaps the
most common use of the feature is to \texttt{include} one file (the ``header'')
within another. By defining the interface in the header, the implementation is
kept inside a single source file, and any number of other source files may use
the interface's implementation. While the macro preprocessor is a necessary
part of the language, it is inherently dangerous. The preprocessor works by
modifying the actual text of the program, which can lead to syntax errors or
subtle bugs that are difficult to solve. Another use of the preprocessor is
for conditional compilation, where different code should be compiled depending
on various conditions such as the underlying architecture or character-encoding
support. Stroustrup advises that the C++ preprocessor should only be used for
these cases~\cite{stroustrup2013the}.

C++'s memory management is similar in some respects to its ancestor, C. Instead
of requesting and returning blocks of memory from the free store by use of
\texttt{malloc()} and \texttt{free()}, a C++ programmer uses \texttt{new} and
\texttt{delete}\footnote{C++ has separate operators for array allocation:
    \texttt{new[]} and \texttt{delete[]}.}. The language inherits the
requirement that the programmer must remember to \texttt{delete} all memory
acquired by \texttt{new} to avoid memory leaks, and that the programmer should
refrain from using deallocated or unallocated memory to prevent invoking
undefined behavior~\cite{stroustrup2013the}. Undefined behavior is
characterized by the \textit{lack} of a definition of what an implementation of
C++ should do in a given situation~\cite{iso/iec}. In other words, if a program
incurs undefined behavior, then an implementation is in no way required to act
in any particular way. The program may not compile, may crash at runtime, or
act in any number of ways. Clearly it is impossible to reason about the
correctness of a program that causes undefined behavior, so C++ programmers
must avoid it at all costs.

C++ programmers are well-aware of the problems that arise from using
\texttt{new} and \texttt{delete} improperly. In fact, Stroustrup warns that
``naked \texttt{new}'' (using \texttt{new} to allocate an object directly)
ought to be avoided. Instead, he advises programmers to use stack-based
allocation when possible, and in other cases use manager objects such as
\texttt{unique\_ptr} and \texttt{shared\_ptr}\footnote{These containers were
    introduced in C++11.}. This idiom is known as ``Resource Acquisition Is
Initialization'', or RAII~\cite{stroustrup2013the}. These containers help
abstract memory management away from the programmer by ensuring \texttt{delete}
is called when the pointer goes out of scope.

\section{D}

D was created in 2001 by Walter Bright. From the name, it is easy to see that D
has drawn much inspiration from C++, as C++ did from C. However, D has made a
number of backwards-incompatible changes from C++ in the process of achieving
its design goals. The D website cites a number of reasons for why D is
necessary. Most relevant to this paper are its assertions about the inherent
complexity in C++ due to the sheer number of features present in the language,
the effort needed to explicitly manage memory, the difficulty in tracking down
pointer bugs, and the hindrance of backwards compatibility with
C~\cite{Doverview}.

D handles memory management through its garbage collector. Classes are
automatically allocated on the heap, and all other data is created on the
stack. The garbage collector frees any memory that has gone out of scope
(though not necessarily immediately after). This removes the need for the
programmer to explicitly allocate memory. If the programmer wishes to obtain
RAII behavior, D provides syntax to execute code after variables go out of
scope through destructors and scope guards~\cite{cppford}.

D also has an answer for the C++ preprocessor. The language contains a number
of features that obviate its need. For example, inclusion is handled by
modules, that handles imports of symbols from other files. This also removes
the need for include guards, because each symbol is only guaranteed to be
imported once. Conditional compilation is achieved through the
\texttt{version} keyword. D also contains an aggressive inliner, which removes
the need to implement small functions as macros~\cite{pretod}.

\section{Rust}

Rust is a quite new programming language developed primarily by Mozilla
starting in 2012. Rust aims to give programmers enough power to access the
computer's hardware while providing ``strong guarantees about isolation,
concurrency, and memory safety''~\cite{Matsakis:2014:RL:2663171.2663188}.
Syntactically, Rust is a more radical departure from C++ than D. However, it
has similar design goals.

One of the interesting features of Rust is its guarantees about heap access.
In Rust, if you were to allocate an int on the heap, you might write

\begin{lstlisting}[language=Rust]
    let x = box 5i;
\end{lstlisting}

This code allocates an This acts similarly to a \texttt{unique\_ptr} would in C++. It guarantees that
you cannot forget to free the memory, and that the memory is not used after
the free. Rust's semantics are more powerful than this, however. Rust
guarantees that ``no other writable pointers alias to this heap memory'',
meaning that it is impossible for multiple objects to write to this memory
(unless the programmer were to use an Rc pointer, which allows multiple
readers)~\cite{RustPointerGuide}.

In lieu of the preprocessor, Rust has an extremely powerful macro system.
Rust macros operate on the abstract syntax tree of the program, rather than on
the text. This allows macros to be type-safe and extend the language itself.
In Rust, \texttt{println} is a macro that performs type-checking on its
arguments to ensure that the format string contains the proper number of
format placeholders for the arguments. In other languages, it might be
possible to check this with a special code path in the compiler, but in Rust
the language itself does the checking.

\nocite{*}
\bibliography{senior-project}
\end{document}
