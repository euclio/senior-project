\documentclass[finalcopy]{srpaper}

\usepackage[toc,page]{appendix}
\usepackage{booktabs}
\usepackage{etoolbox}
\usepackage{hyperref}
\usepackage{listings}
\usepackage{mdframed}
\usepackage[outputdir=latexmk]{minted}
\usepackage{longtable}
\usepackage{ltablex}
\usepackage{url}
\usepackage{verbatimbox}

\definecolor{bg}{rgb}{0.95,0.95,0.95}
\setminted{bgcolor=bg}

\title{Rust vs. D:\ Exploring the Possible Successors of C++}
\author{Andy Russell}
\date{\today}
\advisor{Professor Kim Bruce}

\abstract{The programming languages D and Rust aim to simplify the complex and
error-prone features of C++ while maintaining a similar level of performance.
This paper examines whether the languages succeed in easing the development of
safe code, with a particular focus on each language's compile-time features and
memory management techniques. C++, D, and Rust are evaluated on both subjective
and empirical criteria. In order to evaluate the success of each language's
design goals, I have implemented a number of small programs that demonstrate
common tasks in systems programming, each in C++, D, and Rust. I recruited a
number of volunteers with prior experience with C++ to attempt the
implementation of these programs in D or Rust as well. Each volunteer
documented his or her development process in detail, particularly noting any
errors or bugs that were encountered. The programmers tallied and categorized
each error. This data was used to analyze whether a particular language makes
it easier to avoid certain errors. I then evaluated each language on
expressiveness and ease of development to determine whether the language's
design goals have been met.}

\begin{document}
\frontmatter

\addcontentsline{toc}{section}{List of Listings}
\listoflistings{}

\chapter{Introduction}

Systems programming is an extremely important part of computing today. The raw
speed and low-level access to hardware provided by systems programming
languages are necessary for embedded systems, networking, and gaming, plus any
number of other applications. However, such command of a computer's hardware
naturally invites danger\footnote{Throughout this paper, when I refer to a
concept or feature as ``dangerous,'' I mean that it is error-prone or difficult
to reason about, especially if the errors that may arise from its use are
resistant to debugging.}.

Mistakes in managing memory can lead to run-time crashes or security
vulnerabilities such as buffer overflow or format string
attacks~\cite{Shahriar:2012:MPS:2187671.2187673}. While there are many tools
designed to allow programmers to catch such errors, the ideal solution would be
to eliminate the burden of manual memory management altogether. Other languages
such as Java and C\#, inspired by this goal, have removed that need. However,
due to their dependence on a virtual machine, they have sacrificed performance
and the ability to interface directly with
hardware~\cite{Alexandrescu:2010:DPL:1875434}. The languages examined in this
paper, D and Rust, do not use a virtual machine, instead opting for native
compilation. The languages aim to match the low-level speed and power of C++
while attempting to make code easier to write both from an expressiveness and
correctness standpoint.

This paper aims to analyze the Rust and D programming languages from as
objective a standpoint as possible. I primarily focus on the features that each
language makes at compile time, with memory-safety and preprocessing feature
taking a particular emphasis, due to the misuse of these features causing
dangerous errors in C++.

\chapter{Background}

In order to adequately discuss the design decisions that the implementors of
Rust and D have made, it is important to determine a context from which we can
compare them. Since both languages aim to occupy the same space as C++, I have
decided to compare the languages using C++ as a reference point. In addition
to considering the historical context that each language developed in, I have
decided to focus on a few important features that are essential to systems
programming.

\section{History}

Though C++, D, and Rust are all related, the languages developed within
different historical contexts.

C++ was developed in the early '80s by Bjarne Stroustrup. Stroustrup wished to
bring high-level features such as classes, strong typing, and default arguments
to C. It attempts to provide the programmer with a one-to-one mapping of
built-in types to the hardware, while also offering flexible abstractions to
allow user-defined types to take advantage of the same facilities available to
the built-ins. Stroustrup himself outlined the design philosophy of C++ in two
points:

\begin{itemize}
\item \textit{Leave no room for a lower-level language below C++}
\item \textit{What you don't use you don't pay for}. This is also known as the
``zero-overhead principle''.
\end{itemize}

These principles serve to keep C++ close to its roots in C while maintaining
the abstractions that make it a high-level language~\cite{stroustrup2013the}.
C++ remains one of the most popular languages in the world.

D was created in 2001 by Walter Bright, and later developed by Digital
Mars~\cite{Doverview}. As the name indicates, D has drawn much inspiration from
C++, as C++ did from C. However, D has made a number of backwards-incompatible
changes from C++ in the process of achieving its design goals. The D website
cites a number of reasons why D is necessary. Most relevant to this paper are
its assertions about the inherent complexity in C++ due to the sheer number of
features present in the language, the burden of explicit memory management, the
difficulty in tracking down pointer bugs, and the hindrance of backwards
compatibility with C~\cite{Doverview}. D, while less popular than C++,
maintains a healthy online presence and is often promoted by C++ gurus such as
Andrei Alexandrescu.

Rust is a quite new programming language developed primarily by Mozilla
starting in 2012. Rust aims to give programmers enough power to access the
computer's hardware while providing ``strong guarantees about isolation,
concurrency, and memory safety''~\cite{Matsakis:2014:RL:2663171.2663188}.
Syntactically, Rust is a more radical departure from C++ than D. However, it
has similar design goals. Despite not having a stable release, Rust generates a
considerable amount of discussion and written code, as evidenced by the over
one million downloads served by \url{crates.io}, Rust's package hosting
website~\cite{Cratesio}.

\section{Preprocessing, Macros, and Conditional Compilation}

The C++ macro preprocessor is an integral part of the language. Perhaps the
most common use of the feature is to \texttt{include} one file (the ``header'')
within another. By defining the interface in the header, the implementation is
kept inside a single source file, and any number of other source files may use
the interface's implementation. While the macro preprocessor is a necessary
part of the language, it is inherently dangerous. The preprocessor works by
modifying the actual text of the program, which can lead to syntax errors or
subtle bugs that are difficult to solve. Another use of the preprocessor is
for conditional compilation, where different code should be compiled depending
on various conditions such as the underlying architecture or character-encoding
support. Stroustrup advises that the C++ preprocessor should only be used for
these cases~\cite{stroustrup2013the}.

% TODO: Provide examples

D provides safer alternatives to the C++ preprocessor, and, when possible,
obviates its need entirely. For example, inclusion is handled by modules, which
handle imports of symbols from other files. This also removes the need for
include guards, because each symbol is only guaranteed to be imported once.
Conditional compilation is achieved through the \texttt{version} keyword. D
also contains an aggressive inliner, which removes the need to implement small
functions as macros~\cite{pretod}.

In lieu of the preprocessor, Rust has an extremely powerful macro system.
Rust macros operate on the abstract syntax tree of the program, rather than on
the text. This allows macros to be type-safe and extend the language itself.
In Rust, \texttt{println} is a macro that performs type-checking on its
arguments to ensure that the format string contains the proper number of
format placeholders for the arguments. In other languages, it might be
possible to check this with a special code path in the compiler, but in Rust
the language itself does the checking.

\section{Memory Management}

Memory management is a key component of systems languages. Many
performance-critical applications depend on the power of being able to manage
memory manually. However, this opens up an entire class of bugs. Some of these
bugs include memory leaks (forgetting to free allocated memory) and
use-after-free (using a pointer to memory that has already been reclaimed by
the memory manager). These bugs are often difficult to track down.

C++'s memory management works similarly to its ancestor, C. In fact, due to to
its commitment to backwards compatibility, C++ even inherits the
\texttt{malloc()} and \texttt{free()} functions, as seen in
Listing~\ref{lst:cmemory}. These functions allow the programmer to request and
return blocks of memory to the free store, or heap. However, C++ programmers
should never use these functions, instead opting for the \texttt{new} and
\texttt{delete} operators. C++ has additional operators for array allocation:
\texttt{new[]} and \texttt{delete[]}. Listing~\ref{lst:cppmemory} uses the
operators to allocate an array. These keywords are slightly safer because the
programmer no longer has to keep track of the exact size of the allocation, the
programmer must still remember to to \texttt{delete} all memory acquired by
\texttt{new} to avoid memory leaks, and refrain from using deallocated or
unallocated memory to prevent invoking undefined
behavior~\cite{stroustrup2013the}. In addition, the correct form of the keyword
must be used, as using the non-array \texttt{delete} may leave the heap in an
inconsistent state.

\begin{listing}[H]
\begin{minted}{c}
/* Allocates 10 elements on the heap. */
int* elements = malloc(sizeof(int) * 10);

/* ... */

/* Release the memory for reuse. */
free(elements);
\end{minted}
\caption{C memory management.}
\label{lst:cmemory}
\end{listing}

\begin{listing}[H]
\begin{minted}{c++}
// Allocates 10 elements on the heap.
int[] elements = new int[10];

// ...

// Release the memory for reuse.
delete[] elements;      // Don't forget the `[]'!
\end{minted}
\caption{Primitive C++ memory management.}
\label{lst:cppmemory}
\end{listing}

Undefined behavior is
characterized by the \textit{lack} of a definition of what an implementation of
C++ should do in a given situation~\cite{iso/iec}. In other words, if a program
incurs undefined behavior, then an implementation is in no way required to act
in any particular way. The program may not compile, may crash at run-time, or
act in any number of ways. Clearly it is impossible to reason about the
correctness of a program that causes undefined behavior, so C++ programmers
must avoid it at all costs.

\begin{listing}
\begin{minted}{c++}
#include <array>
#include <memory>

// Allocates 10 elements on the heap.
std::unique_ptr<int[]> elementsPtr(new int[10]);
// Using std::array avoids explicitly using `new'.
std::array<int, 10> elementsArray;

// ...

// Memory is freed when the unique_ptr and array go out of
// scope.
\end{minted}
\label{lst:cppmodernmemory}
\caption{Modern C++ memory management.}
\end{listing}

C++ programmers are well-aware of the problems that arise from using
\texttt{new} and \texttt{delete} improperly. In fact, Stroustrup warns that
``naked \texttt{new}'' (using \texttt{new} to allocate an object directly)
ought to be avoided. Instead, he advises programmers to use stack-based
allocation when possible, and in other cases use manager objects such as
\texttt{unique\_ptr} and \texttt{shared\_ptr}\footnote{These containers were
    introduced in C++11.}. This idiom is known as ``Resource Acquisition Is
Initialization'', or RAII~\cite{stroustrup2013the}. These containers help
abstract memory management away from the programmer by ensuring \texttt{delete}
is called when the pointer goes out of scope.

D's creators acknowledged the problems with manual memory management and opted
to remove the need for it entirely. D handles memory management through a
garbage collector. D classes are automatically allocated on the heap, and all
other data is created on the stack. The garbage collector frees any memory that
has gone out of scope (though not necessarily immediately after). This removes
the need for the programmer to explicitly allocate and deallocate memory. Like
C++, destructors are executed when variables go out of scope, allowing RAII
behavior. In addition, D provides the ``Scope Guard'' statement, which offers
more fine-grained control over when scope-dependent blocks of code should
execute, seen in Listing~\ref{lst:dscope}. For example, if an exception is
thrown, the code would print \texttt{success} and then \texttt{exited} on their
own lines. If an exception was thrown, the code would print \texttt{failure},
then \texttt{exited}.

\begin{listing}[H]
\begin{minted}{d}
import std.stdio;

try {
    scope(exit) writeln("exited");
    scope(failure) writeln("failure");
    scope(success) writeln("success");
    // Code that may throw an exception...
} catch (Exception e) {}
\end{minted}
\caption{D scope guards.}
\label{lst:dscope}
\end{listing}

D programmers may also opt-out of the garbage collector by marking functions
with the \texttt{@nogc} attribute. Within \texttt{@nogc} functions, the garbage
collector will not run. This feature is intended for situations where the
overhead introduced by the D runtime (which contains the garbage collector) is
unacceptable, such as when implementing a kernel or in performance-critical
applications. However, this introduces a number of problems, as allocating
classes and other types that rely on GC allocation will be no longer possible.
In addition, some parts of the standard library perform heap allocations,
though the goal is to eventually mark all of D's standard library as
\texttt{@nogc}.

Rust, on the other hand, attempts to find a middle ground between the
performance of C++ and the overhead of D. One of the interesting features of
Rust is its guarantees about heap access. It accomplishes this by strictly
enforcing the concept of ownership. Ownership can be seen in C++ by looking at
the names of its pointer types\footnote{That is, \texttt{unique\_ptr},
\texttt{shared\_ptr}, or \texttt{weak\_ptr}}. For example, a
\texttt{unique\_ptr} only allows itself \texttt{unique\_ptr} to point at its
memory. It enforces this by disallowing copying and requiring that reassignment
be done through a move, which transfers ownership of a piece of memory from one
object or scope to another. However, ownership transfer may introduce bugs, as
shown in Listing~\ref{lst:cppuseaftermove}.

Rust avoids this class of bug entirely by guaranteeing at compile time that
moved memory cannot be used. Attempting to compile
Listing~\ref{lst:rustuseaftermove} would result in \texttt{error: use of moved
value: `x`}. Even more powerful than this is that Rust guarantees that ``no
other writable pointers alias to this heap memory'', meaning that it is
impossible for multiple objects to write to the same memory location (unless
the programmer were to use an \texttt{Rc} pointer, which allows multiple
readers and writers through reference counting)~\cite{RustPointerGuide}.

\begin{listing}[h]
\begin{minted}{cpp}
#include <memory>
#include <iostream>

// Allocate an int on the heap
std::unique_ptr<int> movedPtr(new int(10));

// Change the int's owner to a new pointer.
std::unique_ptr<int> ptr = std::move(ptr);

// Attempt to dereference the pointers.
std::cout << *ptr << std::endl;
std::cout << *movedPtr << std::endl;    // Segfault!
\end{minted}
\caption{C++ use of moved value (bug).}
\label{lst:cppuseaftermove}
\end{listing}

\begin{listing}[h]
\begin{minted}{rust}
let x = Box::new(5i);       // Allocate an int on the heap.
let y = x;                  // Change the int's owner to y.
println!("{}", x);          // error: use of moved value: 'x'
\end{minted}
\caption{Rust use of moved value (compilation error).}
\label{lst:rustuseaftermove}
\end{listing}

\chapter{Evaluation Strategy}

\section{Criteria}

One of the most challenging aspects of evaluating languages is deciding the
criteria on which they are evaluated. In order to obtain as comprehensive an
evaluation of Rust and D as possible, I wish to include both qualitative and
quantitative criteria.

AlGhamdi and Urban provide an excellent list of qualitative methodologies on
which I have based my own methodology~\cite{AlGhamdi:1993:CAP:162754.162876}.
Their paper summarizes twelve methodologies, and the factors that may be used
to achieve an apt comparison. The following list summarizes the methodologies
expounded in their paper that I will employ in my project.

\begin{enumerate}
\item Comparison of Philosophy and History

Since both Rust and D occupy a similar domain, the main factor that I will use
to distinguish the two is the ``intention of [the] designers''. While these
languages were created nearly a decade apart, differ in their corporate
affiliations, and have varying development team size, I find these factors less
relevant for my study. I believe that the only factor that deeply affects
programmers using the languages is the design philosophy behind them.

\item The Degree of Permissiveness of the Language

This methodology includes criteria such as the ability of the programmer to
circumvent type-checking, operator overloading, and run-time checking. I am
particularly interested in the memory safety of Rust and D compared to C++.
While these languages will allow a programmer to circumvent the type system or
perform unsafe memory accesses, such practices are discouraged. So, I am
interested in the effectiveness of each language in avoiding the requirement of
such unsafe techniques.

\item Language Contributions to Program Readability

It is often said that code is read far more than it is written. In large
systems that depend on reliability, this is surely true. It follows that if
code is easier to read, it is easier to locate defects or bugs.

\item Language Contributions to Program Reliability

This is perhaps the most important methodology to my project. I am particularly
interested in examining how Rust and D attempt to avoid the common pitfalls
that plague C++ development. These include but are not limited to uninitialized
variables, null pointers, and memory leaks.

\item Data Structuring Facilities

This methodology is also quite important to my study. Rust and D are both
strongly typed, and each language provides a number of ways to inform the
compiler of programmer intent for the usage of various data types. Exploring
the numerous primitive data types, and the ability to construct new types from
those primitives is integral to understanding the power of each language.

\item Control Facilities

I am particularly interested in procedural-level control facilities. This
includes parameter-passing methods, concurrency, and generics/templates.

\item Language Contributions to Program Cost

This methodology explores the various costs involved with writing in a language
under consideration. This involves the cost of learning the language, the cost
of writing a program in the language, and even the cost of compiling a program.
The cost of executing and maintaining a program less relevant to this paper,
but I imagine the maintainence cost is somewhat encompassed by the criteria for
evaluating the languages' readability. While this methodology is particularly
important for new programmers, and becomes less apparent with experience, it is
nonetheless important to consider.

\end{enumerate}

In contrast to the methodologies listed above, I found a number of
methodologies irrelevant. For example, D and Rust appear almost identical in
terms of modularity (``Language Contributions to Program Modularity''),
portability (``Portability of a Language''), I/O facilities (``Input/Output''),
and support for inline assembly and foreign functions (``Escape from a
Language''). These methodologies either contribute little to my goal of
studying the memory safety of Rust and D, or the features provided by each
language are so similar that there is little comparison to be made between the
two.

\section{Experimental Design}

While I believe that the aforementioned methodologies comprise a satisfactory
qualitative evaluation strategy for Rust and D, I do not believe that a full
comparison can be achieved with these methodologies alone. Furthermore, many of
the criteria are rather subjective. For my project, I have strived to develop a
method which can be used to compare various language features while avoiding
subjectivity or bias.

Originally, I planned to come up with a number of programs that embodied the
core of systems programming, and then develop these programs in C++, D, and
Rust. Then, using my own experiences, I would attempt to evaluate the
effectiveness of each language for developing these programs. However, this
approach is flawed in a number of ways. For example, suppose I had attempted to
implement a program with logic that required a large amount of conditionals in
Rust. I would likely have encountered a number of bugs with my initial
implementation but eventually have come up with a satisfactory result. Then,
upon moving onto D, I would have avoided most of the errors that I encountered
while working on the Rust implementation. My development process in D would
likely have felt more natural, and the code would probably look eerily like
Rust. To mitigate this problem, I instead recruited a number of volunteers to
learn each language individually, and implement a series of programs meant to
demonstrate various features that are important to systems languages.

I advertised my study both on the Computer Science Facebook group and through
Computer Science colloquium. Once volunteers indicated their interest, I sent
them a document\footnote{The document itself may be found in
Appendix~\ref{app:resources}.}. detailing my expectations for the project. The
volunteers then sent me information including their name, major, experience
with programming, and a confirmation that they read the document in its
entirety.

Each volunteer was then assigned either Rust or D as a language. I tried to
satisfy the volunteer's preference for learning the language, though I first
ensured that the experience level between each language would be comparable.
Over the next month, each volunteer was expected to attempt to implement one
of five programs. The volunteers were not expected to spend more than 2 hours
on each program, but most volunteers worked on each program to completion.

The programs implemented by the volunteers were designed to cover an
assortment of language features as well as to highlight common bugs. The
programs were as follows:

\begin{enumerate}

\item Sentence Splitter

Volunteers were asked to implement a single string into sentences using a set
of heuristics. The difficulty of this program stems from the fact that
while sentences are delimited by periods, question marks, and exclamation
points, they may also include websites, titles, and abbreviations, all of
which are \textit{not} boundaries.

This program was meant to introduce the programmers to string manipulation
techniques and required a large amount of logical branching.

\item Integer Linked List

The next program was to implement a simple linked list data structure that
operates on integers. A number of operations were required to be supported,
including insertion, deletion, retrieval, and methods to query the size, head
and tail of the list.

This program introduced the volunteers to object-orientation, pointer
manipulation, and memory allocation.

\item Generic Array List

The next program involved creating an array list that supported generic
elements. The array list was required to support the same operations as the
integer linked list, and covered the same features of the languages.

\item Parallel Mergesort

Volunteers were then asked to implement the parallel mergesort algorithm. This
algorithm is relatively simple to understand, and easily parallelizable.

This program introduced the programmers to recursion and concurrency.

\item Brainfuck Interpreter

Lastly, the programmers were asked to implement a Brainfuck interpreter.
Brainfuck is an esoteric programming language, known for being fiendishly
unreadable. However, the language's semantics are quite easy to understand. In
short, there are eight meaningful characters in the language, all of which
either manipulate the ``data pointer'' in some way (by incrementing,
decrementing, etc.), or perform I/O on the byte pointed at by the pointer.

This program served as a kind of ``capstone'' for the study. It is larger than
the other programs, and involves file I/O and pointer manipulation.

\end{enumerate}

Each volunteer was also required to document their development process. While
developing each program, the volunteer would note every error encountered,
categorize it as a syntax error, logic error, or resource error, and whether
the error was caught at compile-time or run-time.

\chapter{Results}

Given Rust's focus on compile-time error catching and D's focus on ease of
use, I hypothesized that my volunteers would encounter a much greater number
of compile time issues with Rust, and that D programmers would find that their
programs ran into more runtime issues. Also, D's syntax would be more familiar
to my volunteers. I found that my hypothesis was true.

\section{Experimental Results}

I was able to recruit seven volunteers that generously donated their time and
effort to my project. Each volunteer was very experienced in programming: six
were computer science majors, and one was a computer science minor.
Figure~\ref{fig:sampleerrorlog} contains a snippet of an error log submitted
by one of my volunteers.

\begin{figure}[h]
\begin{lstlisting}[breaklines]
Syntax error - Wrong syntax of println - Compile time
Syntax error - Wrote "splitSentences" instead of "split_sentences" - Compile time
Syntax error - Used old variable name - Compile time
Syntax error - Tried to iterate over len() instead of range(0, len()) - Compile time
Syntax error - Used one argument instead of two with range() - Compile time
Syntax error - Passed char instead of &char in the contains func - Compile time
Syntax error - Used [char] instead of &[char] in initialization of array - Compile time
\end{lstlisting}
\caption{Sample error log}
\label{fig:sampleerrorlog}
\end{figure}

All of my volunteers were able to complete the sentence splitter. It was very
interesting to see the types of errors that were encountered during this phase
of the experiment. As expected, Rust programmers found many more compile time
issues. Surprisingly, D programmers ran into segmentation faults due to the
behavior of attempting to access iterators out of bounds.

\begin{table}[h]
\centering
\begin{tabular}{crrrl}
\toprule
Volunteer & Syntax & Logic & Resource & Notes \\
\midrule
A & 6 & 2 & 0 & \\
B & 5 & 0 & 1 & Did not finish \\
C & 8 & 2 & 0 & Test suite passes \\
D & 2 & 1 & 0 & Program enters infinite loop on ellipsis \\
E & 18 & 5 & 0 & Test suite passes \\
F & 15 & 0 & 0 & Passes \\
G & 4 & 2 & 0 & Passes \\
\bottomrule
\end{tabular}
\caption{Sentence splitter error log summary}
\label{tab:sentencesplitter}
\end{table}

\section{Evaluation}

\chapter{Conclusion}

\nocite{*}
\bibliography{senior-project}

\appendix
\chapter{Volunteer Resources}\label{app:resources}
The following documents were sent to the volunteers of my project. The
documents contain explanations of the project expectations, brief descriptions
of the programs to be implemented, and installation instructions for the
languages themselves.

The documents were written using Markdown. The versions included in this
appendix are rendered from \LaTeX{} generated from the Markdown source. The
documents appear online at
\url{https://github.com/euclio/senior-project/tree/master/resources}.

\clearpage
\section{Project Volunteer
Information}\label{project-volunteer-information}

Thank you for volunteering your time to help me with my senior project.
In this document I will attempt to cover my expectations from you as
well as any resources that you might find useful while working on my
project. Please let me know if you have any questions; I'd be happy to
discuss the project with you!

\subsection{Expectations}\label{expectations}

Over the course of this project, you will be implementing a series of
programs that illustrate various aspects of systems programming: such as
memory management, pointer manipulation, and conditional compilation.
Each volunteer is expected to have

\begin{itemize}
\itemsep1pt\parskip0pt\parsep0pt
\item
  Some experience with C++
\item
  Very little experience with either D or Rust
\end{itemize}

\subsubsection{Error Logs}\label{error-logs}

For my project, I am interested in the types of errors that are common
to each of the languages you will be working in. Specifically, here are
the types of errors I am studying:

\begin{center}
\begin{tabular}[c]{@{}lp{9cm}@{}}
\toprule
Category & Description \\
\midrule
Syntax error & Any error caused by your program failing to parse. Could
be a typo in a variable name, forgetting to put braces,
etc.\tabularnewline
Logic error & Your program parses correctly, but it does not run as
intended due to human error. Off-by-one errors are a common example of a
logic error.\tabularnewline
Resource error & The incorrect use of memory management. For example,
your program dereferences a \lstinline!NULL! pointer or uses deallocated
memory.\tabularnewline
\bottomrule
\end{tabular}
\end{center}

In order to obtain data on the types of errors that you run into, I
would like you to keep a log of every error you run into while working
on each project. Each log should contain the type of error you ran into,
a brief description of the error, and whether the error was caught at
compile-time or run-time. Here's an example of a log:

\begin{longtable}[c]{@{}lll@{}}
\toprule
Category & Description & When caught\tabularnewline
\midrule
\endhead
Syntax error & Wrote ``sring'' instead of ``string'' &
Compile-time\tabularnewline
Logic error & Accidentally iterated one time too many &
Run-time\tabularnewline
Resource error & Used a freed pointer & Run-time\tabularnewline
\bottomrule
\end{longtable}

In short, if the compiler spits out an error, or you have to change your
code after you noticed something didn't work, log it! The error logs are
key to my study so please take care that they are accurate.

\paragraph{Rust Note}\label{rust-note}

Since Rust is rather unstable, you might receive warnings about
deprecated or unstable features. You may fix these, and you do
\emph{not} have to include them in your logs. However, if you are unsure
if an error should be included or not, please include it anyways.

\subsubsection{Implementation}\label{implementation}

Each volunteer will be selected to implement their series of programs in
D or Rust.

\paragraph{Resources}\label{resources}

In order to keep the learning environment as free of bias as possible,
the resources you may use to learn Rust and D should be limited to the
following sites (i.e., no StackOverflow, no IRC, etc.):

\subparagraph{Additional Help}\label{additional-help}

If you would like additional help, you may contact me in which case I
will decide to offer help at my own discretion. I will offer help with:

\begin{itemize}
\itemsep1pt\parskip0pt\parsep0pt
\item
  Algorithmic Questions
\item
  Clarification
\item
  Installing the languages
\item
  Locating information about the languages in the books or docs
\end{itemize}

I will not:

\begin{itemize}
\itemsep1pt\parskip0pt\parsep0pt
\item
  Look at your code
\item
  Help you solve compile errors
\end{itemize}

You are free to use whatever editor you wish. Please install a syntax
highlighting package for your editor; I know that packages exist for
Sublime Text, emacs, and vim. Please do \emph{not} install any
autocomplete packages, as these may skew the results of the study.

\paragraph{Development}\label{development}

There will be test cases provided in order for you to test your
implementation. However, there is no problem if your program does not
pass all cases. As long as you spend a significant amount of time on
each program (say, 90 minutes), it doesn't matter if your program
``works'' in the end.

I may use snippets of your code in my final paper, in which case your
work will be kept anonymous.

\subsection{Installation}\label{installation}

It's probably easiest to use your own computer to install each
language's compiler. However, if you have a Windows PC, to save yourself
some installation headache I recommend either dual-booting or running a
VM of a Linux distro such as Ubuntu, or using the lab Macs. I've set up
\lstinline!project! with the needed dependencies of each language as
well.

You may find language-specific installation guides in the
\href{./rust_resources.md}{Rust resources} and \href{./d_resources.md}{D
resources} documents.

\subsection{Programming}\label{programming}

You will be asked to implement the following programs (listed in order
from easiest to hardest).

\emph{Note:} While I don't expect you to handle \emph{all} errors that
your program might run into, if you run into exceptional conditions
(that is, empty input, invalid indices, etc.), please handle the errors
as you feel fit. This could include throwing an exception or returning
an \lstinline!Error! object.

\subsubsection{\texorpdfstring{Sentence Splitter
(\href{http://www.ling.gu.se/~lager/python_exercises.html}{source})}{Sentence Splitter (source)}}\label{sentence-splitter-sourcesentence-splitter}

Define a function capable of splitting a text into sentences. The
standard set of heuristics for sentence splitting includes (but isn't
limited to) the following rules:

Sentence boundaries occur at one of ``.'' (periods), ``?'' or ``!'',
except that

\begin{itemize}
\itemsep1pt\parskip0pt\parsep0pt
\item
  Periods followed by whitespace followed by a lower case letter are not
  sentence boundaries.
\item
  Periods followed by a digit with no intervening whitespace are not
  sentence boundaries.
\item
  Periods followed by whitespace and then an upper case letter, but
  preceded by any of a short list of titles are not sentence boundaries.
  Sample titles include Mr., Mrs., Dr., Jr.
\item
  Periods internal to a sequence of letters with no adjacent whitespace
  are not sentence boundaries (for example, www.aptex.com, or e.g).
\item
  Periods followed by certain kinds of punctuation (notably comma and
  more periods) are probably not sentence boundaries.
\end{itemize}

Write a function \lstinline!splitSentences(string input)! that, given a
string, prints each sentence on a separate line. I am only concerned
with the function, so you may pass your test input to the function in
any way, whether it be hard-coded, or through stdin, etc.

\emph{Note:} For simplicity, we will assume that we only have one space
between each sentence.

\paragraph{Skills covered}\label{skills-covered}

\begin{itemize}
\itemsep1pt\parskip0pt\parsep0pt
\item
  String manipulation
\item
  Branching
\end{itemize}

\subsubsection{Integer Linked List}\label{integer-linked-list}

Define a
\href{http://en.wikipedia.org/wiki/Linked_list\#Singly_linked_list}{linked
list} data structure that operates on integers. The linked list should
support:

\begin{itemize}
\itemsep1pt\parskip0pt\parsep0pt
\item
  Insertion in O(n) time.
\item
  Deletion in O(n) time.
\item
  Retrieval in O(n) time.
\item
  A method to retrieve the size.
\item
  A method to retreive the head of the list.
\item
  A method to retrieve the tail of the list.
\end{itemize}

The IntegerLinkedList class should implement the following Java-like
interface:

\begin{lstlisting}[language=Java]
/**
 * A linked list containing only integers.
 */
interface IntegerLinkedList {
  /**
   * Inserts an element into the linked list at the specified index,
   * shifting any elements already in its position down.
   */
  public void insert(int index, int element);

  /**
   * Removes an element from the specified index, shifting any elements
   * already in the list back up.
   * @return The element that was removed.
   */
  public int remove(int index);

  /**
   * Retrieves an element at the specified position
   */
  public int get(int index);

  /**
   * Returns the number of integers in the linked list.
   */
  public int size();

  /**
   * Returns the first element of the linked list.
   */
  public int head();

  /**
   * Returns the last element of the linked list.
   */
  public int tail();
}
\end{lstlisting}

\paragraph{Skills covered}\label{skills-covered-1}

\begin{itemize}
\itemsep1pt\parskip0pt\parsep0pt
\item
  Object-orientation
\item
  Pointer manipulation
\item
  Memory allocation
\end{itemize}

\subsubsection{Generic Array List}\label{generic-array-list}

Implement an array list class (also known as a dynamic array). Unlike
the linked list class, this class should support generic elements, like
ArrayList in Java. You should use templates (D) or generics (Rust) to
implement this class.

The array list should support:

\begin{itemize}
\itemsep1pt\parskip0pt\parsep0pt
\item
  Insertion in amortized O(1) time.
\item
  Deletion in O(n) time.
\item
  Retrieval in O(1) time.
\item
  A method to retrieve the size.
\end{itemize}

The ArrayList class should implement the following Java-like interface:

\begin{lstlisting}[language=Java]
/**
 * An array that dynamically resizes as elements are added.
 */
interface ArrayList<E> {
  /**
   * Inserts an element into the array at the specified index,
   * shifting any elements already in the list down.
   */
  public void insert(int index, E element);

  /**
   * Removes an element from the specified index, shifting any
   * elements past its position back up.
   * @return The element that was removed.
   */
  public E remove(int index);

  /**
   * Retrieves an element at the specified index.
   */
  public E get(int index);

  /**
   * Returns the number of elements in the array.
   */
  public int size();
}
\end{lstlisting}

\paragraph{Skills covered}\label{skills-covered-2}

\begin{itemize}
\itemsep1pt\parskip0pt\parsep0pt
\item
  Object-orientation
\item
  Generic programming
\item
  Memory allocation
\end{itemize}

\subsubsection{Parallel Mergesort}\label{parallel-mergesort}

Implement the
\href{http://en.wikipedia.org/wiki/Merge_sort\#Parallel_merge_sort}{parallel
mergesort algorithm}. This algorithm should use
\href{http://dlang.org/phobos/std_parallelism.html}{std.parallelism} (D)
or \href{http://doc.rust-lang.org/std/thread/}{std::thread} (Rust). You
should also implement a sequential threshold of 10, meaning that if the
number of elements in a subarray is less than 10, you should not fork a
new thread.

Your function should take in an array.

\paragraph{Skills covered}\label{skills-covered-3}

\begin{itemize}
\itemsep1pt\parskip0pt\parsep0pt
\item
  Recursion
\item
  Concurrency
\end{itemize}

\subsubsection{Brainfsck Interpreter}\label{brainfsck-interpreter}

Write an interpreter for the
\href{http://esolangs.org/wiki/Brainfuck}{brainfsck} programming
language (thankfully, you are not required to write any brainfsck
programs on your own).

Your interpreter should take in a file name as the first argument to the
program, and respond to standard input and output as specified in
\href{http://en.wikipedia.org/wiki/Brainfuck\#Commands}{this Wikipedia
article}. Hint: your interpreter may need to perform multiple passes on
the input to handle brackets.

\paragraph{Skills covered}\label{skills-covered-4}

\begin{itemize}
\itemsep1pt\parskip0pt\parsep0pt
\item
  File I/O
\item
  Pointer manipulation
\end{itemize}

\subsection{Completion}\label{completion}

Upon completion of each program, you may email your code and error log
to me or upload it to a source-hosting website such as GitHub or
BitBucket.

\subsection{Conclusion}\label{conclusion}

Thank you again for volunteering! If you would like to contact me, I may
be reached on Facebook, through email at acr02011@mymail.pomona.edu, or
through my phone at (314) 440-8830.

\clearpage
\section{D Resources}\label{d-resources}

\subsection{Installation}\label{installation}

\href{http://dlang.org}{D} has a number of compiler implementations, but
we will be using \lstinline!dmd!, the reference compiler.

On Windows or Macs, you may install D by using the
\href{http://ftp.digitalmars.com/dinstaller.exe}{binary installer}.

On Linux, you can likely install D with your distro's package manager,
but it might not be in the official repositories. For example, on
Ubuntu, you will have to add the d-apt repository and install
\lstinline!dmd!.

\begin{lstlisting}[language=sh]
$ sudo wget \
    http://master.dl.sourceforge.net/project/d-apt/files/d-apt.list \
    -O /etc/apt/sources.list.d/d-apt.list
$ sudo apt-get update
$ sudo apt-get -y --allow-unauthenticated install \
    --reinstall d-apt-keyring && sudo apt-get update
$ sudo apt-get install dmd
\end{lstlisting}

If you're using \lstinline!project!, you'll have to build D from source.
I've written a \href{./install_d.sh}{shell script} to do this, as the
process is pretty involved.

\subsection{D Documents}\label{d-documents}

\begin{itemize}
\itemsep1pt\parskip0pt\parsep0pt
\item
  \href{http://dlang.org/intro.html}{D Reference}
\item
  \href{http://ddili.org/ders/d.en/index.html}{Programming in D}
\end{itemize}

You should read the following resources before getting started with D,
though you will likely find other sections helpful:

\begin{itemize}
\itemsep1pt\parskip0pt\parsep0pt
\item
  \href{http://ddili.org/ders/d.en/index.html}{All Chapters up to
  Strings}
\item
  \href{http://ddili.org/ders/d.en/lifetimes.html}{Lifetimes}
\item
  \href{http://ddili.org/ders/d.en/member_functions.html}{Member
  Functions}
\item
  \href{http://ddili.org/ders/d.en/pointers.html}{Pointers}
\item
  \href{http://ddili.org/ders/d.en/templates.html}{Templates}
\item
  \href{http://ddili.org/ders/d.en/parallelism.html}{Parallelism}
\item
  \href{http://ddili.org/ders/d.en/files.html}{Files}
\end{itemize}

\clearpage
\section{Rust Resources}\label{rust-resources}

\subsection{Installation}\label{installation}

\href{https://github.com/rust-lang/rust}{Rust} is a very volatile
language, with breaking changes being brought into the master branch
almost every day. Most developers who use Rust use the nightly releases
which means that they must constantly update their code to deal with the
most recent breaking changes. Since I do not want to introduce any
additional overhead to the project, we will be sticking to the
\textbf{1.0.0-alpha.2} release of Rust. Please ensure that you use this
version in order to keep source compatibility during your time working
on the project.

You may install Rust using one of the 1.0.0-alpha.2 (\textbf{not
nightly}) binary installers
\href{http://www.rust-lang.org/install.html}{located here}.

If you have problems with the binary installers, you may install Rust
from source in the following way. If you are attempting to build Rust on
\lstinline!project!, you may skip step 1, as the dependencies are
already installed.

\begin{enumerate}
\def\labelenumi{\arabic{enumi}.}
\item
  Install dependencies

  Use homebrew or your distro's package manager to install

  \begin{itemize}
  \itemsep1pt\parskip0pt\parsep0pt
  \item
    \lstinline!g++! 4.7 or \lstinline!clang++! 3.7
  \item
    \lstinline!python! 2.6 or later (not 3.x)
  \item
    GNU \lstinline!make! 3.81 or later
  \item
    \lstinline!curl!
  \item
    \lstinline!git!
  \end{itemize}
\item
  Download and build Rust:

\begin{lstlisting}[language=sh]
$ git clone https://github.com/rust-lang/rust.git && cd rust
$ git checkout 1.0.0-alpha.2
$ ./configure --prefix=$PWD
$ make -j5 && make install
\end{lstlisting}

  This will take a while, on my laptop it took just over an hour to
  compile.
\item
  Add Rust to your path.

\begin{lstlisting}[language=sh]
echo "export PATH=\$PATH:$PWD/bin" >> ~/.bashrc
source ~/.bashrc
\end{lstlisting}

  Test that rust installed correctly.

\begin{lstlisting}[language=sh]
$ rustc --version
rustc 1.0.0-dev
\end{lstlisting}
\end{enumerate}

\subsection{Documentation}\label{documentation}

Note that he docs are pinned to the 1.0.0-alpha.2 release.

\begin{itemize}
\itemsep1pt\parskip0pt\parsep0pt
\item
  \href{http://doc.rust-lang.org/1.0.0-alpha.2/index.html}{Rust
  Documentation}
\item
  \href{http://rustbyexample.com/}{Rust by Example}
\end{itemize}

\emph{Note:} since Rust is changing so rapidly, there is a chance that
Rust by Example will be outdated. The official documentation and
associated book will be correct, however.

You should read the following resources before getting starting with
Rust, though you will likely find other sections helpful:

\begin{itemize}
\itemsep1pt\parskip0pt\parsep0pt
\item
  \href{http://doc.rust-lang.org/1.0.0-alpha.2/book/basic.html}{All of
  Chapter 2: Basics} (skip the installation chapter)
\item
  \href{http://doc.rust-lang.org/1.0.0-alpha.2/book/method-syntax.html}{Method
  Syntax} (classes)
\item
  \href{http://doc.rust-lang.org/1.0.0-alpha.2/book/pointers.html}{Pointers}
  and
  \href{http://doc.rust-lang.org/1.0.0-alpha.2/book/ownership.html}{Ownership}
\item
  \href{http://doc.rust-lang.org/1.0.0-alpha.2/book/generics.html}{Generics}
\item
  \href{http://doc.rust-lang.org/1.0.0-alpha.2/book/concurrency.html}{Concurrency}
\item
  \href{http://rustbyexample.com/file.html}{File I/O}
\end{itemize}


\chapter{Solutions}\label{app:solutions}
This appendix includes my solutions to the problems that were given to my
volunteers. The code in each example strives to be as idiomatic as possible in
each language.

\setminted{bgcolor=}

\section{Hello, World}
\subsection{C++}
\begin{mdframed}[linecolor=black]
\inputminted{cpp}{../examples/hello-world/hello_world.cpp}
\end{mdframed}

\subsection{D}
\begin{mdframed}[linecolor=black]
\inputminted{d}{../examples/hello-world/hello_world.d}
\end{mdframed}

\subsection{Rust}
\begin{mdframed}[linecolor=black]
\inputminted{rust}{../examples/hello-world/hello_world.rs}
\end{mdframed}

\section{Sentence Splitter}
\subsection{C++}
\begin{mdframed}[linecolor=black]
\inputminted[fontsize=\scriptsize]{cpp}{../examples/sentence-splitter/sentence_splitter.cpp}
\end{mdframed}

\subsection{D}
\begin{mdframed}[linecolor=black]
\inputminted[fontsize=\scriptsize]{d}{../examples/sentence-splitter/sentence_splitter.d}
\end{mdframed}

\subsection{Rust}
\begin{mdframed}[linecolor=black]
\inputminted[fontsize=\scriptsize]{rust}{../examples/sentence-splitter/sentence_splitter.rs}
\end{mdframed}

\section{Integer Linked List}
\subsection{C++}
\begin{mdframed}[linecolor=black]
\inputminted[fontsize=\scriptsize]{cpp}{../examples/int-linked-list/int_linked_list.hpp}
\end{mdframed}

\begin{mdframed}[linecolor=black]
\inputminted[fontsize=\scriptsize]{cpp}{../examples/int-linked-list/int_linked_list.cpp}
\end{mdframed}

\subsection{D}
\begin{mdframed}[linecolor=black]
\inputminted[fontsize=\scriptsize]{d}{../examples/int-linked-list/int_linked_list.d}
\end{mdframed}

\subsubsection{Test Case}
\begin{mdframed}[linecolor=black]
\inputminted[fontsize=\scriptsize]{d}{../examples/int-linked-list/tests/int-linked-list-test.d}
\end{mdframed}

\subsection{Rust}
\begin{mdframed}[linecolor=black]
\inputminted[fontsize=\scriptsize]{rust}{../examples/int-linked-list/int_linked_list.rs}
\end{mdframed}

\subsubsection{Test Case}
\begin{mdframed}[linecolor=black]
\inputminted[fontsize=\scriptsize]{rust}{../examples/int-linked-list/tests/int-linked-list-test.rs}
\end{mdframed}

\section{Generic Array List}
\subsection{C++}
\begin{mdframed}[linecolor=black]
\inputminted[fontsize=\scriptsize]{cpp}{../examples/generic-array-list/generic_array_list.hpp}
\end{mdframed}

\begin{mdframed}[linecolor=black]
\inputminted[fontsize=\scriptsize]{cpp}{../examples/generic-array-list/generic_array_list.cpp}
\end{mdframed}

\subsection{D}
\begin{mdframed}[linecolor=black]
\inputminted[fontsize=\scriptsize]{d}{../examples/generic-array-list/generic_array_list.d}
\end{mdframed}

\subsubsection{Test Case}
\begin{mdframed}[linecolor=black]
\inputminted[fontsize=\scriptsize]{d}{../examples/generic-array-list/tests/generic_array_list_test.d}
\end{mdframed}

\subsection{Rust}
\begin{mdframed}[linecolor=black]
\inputminted[fontsize=\scriptsize]{rust}{../examples/generic-array-list/generic_array_list.rs}
\end{mdframed}

\subsubsection{Test Case}
\begin{mdframed}[linecolor=black]
\inputminted[fontsize=\scriptsize]{rust}{../examples/generic-array-list/tests/generic-array-list-test.rs}
\end{mdframed}

\section{Parallel Merge Sort}
\subsection{C++}
\begin{mdframed}[linecolor=black]
\inputminted[fontsize=\scriptsize]{cpp}{../examples/parallel-merge-sort/parallel_merge_sort.cpp}
\end{mdframed}

\subsection{D}
\begin{mdframed}[linecolor=black]
\inputminted[fontsize=\scriptsize]{d}{../examples/parallel-merge-sort/parallel_merge_sort.d}
\end{mdframed}

\subsection{Rust}
\begin{mdframed}[linecolor=black]
\inputminted[fontsize=\scriptsize]{rust}{../examples/parallel-merge-sort/parallel_merge_sort.rs}
\end{mdframed}

\section{Brainfuck Interpreter}

\subsection{C++}
\begin{mdframed}[linecolor=black]
\inputminted[fontsize=\scriptsize]{cpp}{../examples/brainfsck/brainfsck.cpp}
\end{mdframed}

\subsection{D}
\begin{mdframed}[linecolor=black]
\inputminted[fontsize=\scriptsize]{d}{../examples/brainfsck/brainfsck.d}
\end{mdframed}

\subsection{Rust}
\begin{mdframed}[linecolor=black]
\inputminted[fontsize=\scriptsize]{rust}{../examples/brainfsck/brainfsck.rs}
\end{mdframed}

\subsection{Brainfuck Test Case (``Hello World'')}
\begin{mdframed}[linecolor=black]
\inputminted[fontsize=\scriptsize]{brainfuck}{../examples/brainfsck/tests/hello.bf}
\end{mdframed}

\end{document}
